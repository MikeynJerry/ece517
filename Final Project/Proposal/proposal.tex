\documentclass[titlepage]{article}
\usepackage[utf8]{inputenc}
\usepackage{graphicx,verbatim,amsmath,amssymb,amsthm}
\usepackage{subcaption}
\graphicspath{ {./images/} }

\title{Implementing Deep Q Learning and Double Deep Q learning in training Pac-Man \\ 
\ \\
\large {Final Project: Project Proposal}\\
\ \\
ECE 517 - Reinforcement Learning \\
Prof. Dr. Amir Sadovnik
}

\author{Jerry Duncan, Fabian Fallas, Tabitha Samuel }
\date{November 2020}

\begin{document}

\maketitle
\section{Problem definition}
Using reinforcement learning techniques to teach algorithms to 'play' Atari games has been a popular and constantly evolving research area in the past five years. For this final project, we propose to use Deep Q learning, and double deep Q learning reinforcement learning techniques to train a model to successfully play the Pac-Man game. We will initially train the models on the small grid provided by the \cite{openaigym} , and then compare change in complexity and therefore runtime when they are run on the medium sized grid. 

\section{Problem motivation and background}
We propose to build upon the work described in \cite{gnanasekaranreinforcement}. In this paper, the group investigates the effectiveness of Deep Q-learning based on the context of the Pac-man game using Q-learning and Approximate Q-learning as baselines. We plan on implementing deep Q learning, and adding an additional model with Double Deep Q learning, and compare performance and trade offs between the two techniques. We will use the code base from \cite{pacmandqn} as a baseline code that we will extend to include Double Deep Q-Learning.

\section{Evaluation of results}
We would like to evaluate results from three perspectives in this project:
\begin{itemize}
\item Evaluate increase in complexity and therefore training time when we  move from the small grid to the medium sized grid.
\item Compare performance of Deep Q-learning versus Double Deep Q-learning
\item What else?
\end{itemize}

\bibliographystyle{IEEEtran}
\bibliography{proposal}
\end{document}


